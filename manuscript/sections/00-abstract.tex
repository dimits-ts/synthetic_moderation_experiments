% !TEX root = ../main.tex
%

\begin{abstract}
    Despite the increasingly significant role of online moderation, there have been limited large-scale studies evaluating alternative moderation strategies. This is largely constrained by the absence of suitable datasets and the difficulty in involving human participants, discussants, moderators, and evaluators in multiple experiments. To address these limitations, we propose a methodology utilizing synthetic experiments performed exclusively by \acp{LLM} to initially bypass the need for human involvement. We use this methodology to evaluate multiple current moderation approaches and baselines, and propose a novel \ac{LLM} moderation prompt, which outperforms current alternatives. We also investigate how different aspects of our methodology lead to more diverse discussions. Finally, we introduce an efficient, purpose-built, open-source Python framework, named “SynDisco” (\texttt{pip install syndisco}), designed to facilitate large-scale simulation of discussions and provide the “\ac{VMD}”, a comprehensive dataset comprising LLM-generated and LLM-annotated discussions sourced from multiple open-source \acp{LLM} \datasetlink.
\end{abstract}