% !TEX root = ../main.tex
%
\section{Background and Related Work}

\subsection{Synthetic Discussions}
\label{ssec:related:discussions}

While studies exist for simulating user interactions in social media \cite{park_simulacra, mou_2024, tornberg_2023, y_social, balog_2024}, and for using \ac{LLM} facilitators \cite{kim_et_al_chatbot, cho-etal-2024-language}, none so far have combined the two approaches. 


\citet{balog_2024} propose a methodology for generating synthetic discussions by extracting topics and comments from real online ones and prompting an \ac{LLM} to continue them. However, they do not use \ac{LLM}-based user agents to simulate conversational dynamics, nor do they include facilitators in their setup. Their method also struggles with malformed metadata (e.g., missing usernames) generated by the \ac{LLM}, for which they only suggest error detection as a solution. Additionally, their approach depends on the availability of appropriate human discussion datasets.

\citet{ulmer2024} create synthetic discussions between two roles: an agent controlling a fictional environment and a client interacting with it. These discussions are filtered and used to finetune the agent LLM for a specific task. Our methodology generalizes their framework: an agent (facilitator) interacts with multiple clients (non-facilitator users).

Finally, \citet{abdelnabi_negotiations} generate synthetic negotiations involving multiple agents with different agendas and responsibilities. Our work can be seen as a domain shift of their approach --- from negotiation to discussion facilitation --- where various user types (e.g., normal users, trolls, community veterans) engage in discussion moderated by a facilitator with veto power.


\subsection{LLM Facilitation}

Unlike \ac{ML} classification models traditionally used in online platforms, \acp{LLM} can actively facilitate discussions \cite{korre2025evaluation}. They can warn users for rule violations \cite{Kumar_AbuHashem_Durumeric_2024}, monitor engagement \cite{schroeder-etal-2024-fora}, aggregate diverse opinions \cite{small-polis-llm}, and provide translations and writing tips, which is especially useful for marginalized groups \cite{Tsai2024Generative}. These capabilities suggest that \acp{LLM} may be able to assist or even replace human facilitators in many tasks \cite{small-polis-llm, seering_self_moderation}.

Moderator chatbots have shown promise; \citet{kim_et_al_chatbot} demonstrated that simple rule-based models can enhance discussions, although their approach was largely confined to organizing the discussion based on the ``think-pair-share'' framework \cite{ahmad_2010_supporting, Navajas2018}, and balancing user activity. \citet{cho-etal-2024-language} use \ac{LLM} facilitators in human discussions, with facilitation strategies based on Cognitive Behavioral Therapy and the work of \citet{rosenberg2015nonviolent}. They show that \ac{LLM} facilitators can provide “specific and fair feedback” to users, although they struggle to make users more respectful and cooperative.  In contrast to both works, our work uses exclusively \ac{LLM} participants and \ac{LLM} facilitators, and tests the latter in an explicitly toxic and challenging environment.


\subsection{Discussion Quality}
\label{ssec:related:quality}

In this paper we need to evaluate two different quality dimensions. One is \emph{discussion quality as seen by humans}. and the other relates to \emph{``high-quality '' or ``useful'' synthetic data}.

Discussion quality is difficult to measure both because of the breadth of the possible goals of a discussion, as well as the lack of established computational metrics in Social Science literature \cite{korre2025evaluation}. In our study, we will use \emph{toxicity} as a proxy for discussion quality, since it can inhibit online and deliberative discussions \citep{dekock2022disagree, XiaToxicity}\footnote{We note that this is not always true \citep{Avalle2024PersistentIP}.}. We use \acp{LLM} as classification models (\S\ref{ssec:experimental:evaluation}), as they are reliable for toxicity detection \citep{kang-qian-2024-implanting, Wang2022ToxicityDW, anjum2024hate}.


The second quality dimension---“high-quality” or “useful” data---is essential in \ac{LLM}-based discussion frameworks, as such discussions tend to deteriorate quickly without human involvement, often becoming repetitive and low-quality \citep{ulmer2024}. Despite this importance, methods for quantifying the quality of synthetic data remain limited.

 \citet{balog_2024} use a mix of graph-based, methodology-specific, and lexical similarity metrics, many of which depend on human discussion datasets. Their most generalizable measure is a loosely defined “coherence” score, which is \ac{LLM}-annotated without theoretical grounding. \citet{kim_et_al_chatbot} assess quality through post-discussion surveys and by measuring lexical diversity to approximate the variety of opinions expressed. \citet{ulmer2024}  introduce a metric called \emph{``Diversity''}, which penalizes repeated text sequences between comments using ROUGE-L \citep{lin-2004-rouge} scores. This metric is described in further detail in Section \S\ref{ssec:experimental:evaluation}.


\subsection{LLMs as Human Subjects}
\label{ssec:related:human-llm}

\citet{grossman_2023} argue that synthetic agents have the potential to eventually replace human participants, a perspective shared by other researchers \cite{tornberg_2023, argyle2023}. Indeed, \acp{LLM} have demonstrated complex, emergent social behaviors \cite{Park2023GenerativeAI, demarzo_2023, leng_2024, abdelnabi_negotiations, abramski_2023, hewitt2024predicting, park2024generativeagentsimulations1000}.

However, significant limitations of \acp{LLM} remain in the context of Social Science experiments. Issues include undetectable behavioral hallucinations \cite{rossi_2024}; socio-demographic, statistical and political biases \cite{anthis_2025,hewitt2024predicting,rossi_2024, Taubenfeld2024SystematicBI}; unreliable annotations \cite{jansen_2023,bisbee_2023,neumann_2025, Gligoric2024CanUL}; non-deterministic outputs \cite{atil_2025, bisbee_2023}; and excessive agreeableness \cite{Park2023GenerativeAI, anthis_2025, rossi_2024}.

Our study must thus be conservative towards the generalizability of our results to discussions with humans. Reproduction studies with humans are ultimately needed, and we leave them for future work.


%Low diversity points to pathological problems (e.g., \acp{LLM} repeating previous comments). On the other hand, extremely high diversity may point to a lack of interaction between participants; a discussion in which participants engage with each other will feature some lexical overlap (e.g., common terms, paraphrasing points of other participants). Diversity is defined as:

%\small
%\begin{equation}
%	\label{eq:variety}
%	\textit{div}(d) = 1 - \frac{2}{N_d(N_d-1)}
%	\sum_{i=1}^{N_d-1} \sum_{\substack{j=i+1}}^{N_d} R(c(i,d), c(j,d))
%\end{equation}
%\normalsize
%\noindent where \textit{R} is the ROUGE-L F1 score\footnote{We use the \href{https://pypi.org/project/rouge-score}{rouge-score} package in our analysis.} \cite{lin-2004-rouge}, and $N_d$ the length (in comments) of discussion $d$.

%We will be using \emph{diversity} in this paper, although we note that more robust synthetic quality metrics are needed.
