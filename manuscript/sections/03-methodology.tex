% !TEX root = ../main.tex
%

\section{Methodology}
\label{sec:methodology}

\subsection{Defining synthetic discussions}
\label{ssec:methodology:discussions}

Let $U$ be the set of users participating in discussions and $M$ the set of moderators/facilitators, where $M \cap U = \emptyset$. We define a discussion $d$ of $\lvert d \rvert$ comments\footnote{Also referred to as “dialogue turns” in some publications.} $c(d, i)$ as an ordered set:
\begin{equation}
    d = \{c(d, 1), c(d, 2), \ldots\}
\end{equation}


Next, we define a turn-taking function $u: D  \times \mathbb{N} \rightarrow U \cup M$ mapping a comment in the $i$-th turn of a discussion $d \in D$ to an arbitrary user in $U$, or moderator/facilitator in $M$. In real discussions, $u$ is not strictly defined, since which user responds to each comment can not be reliably determined. However, in a synthetic environment, $u$ can be made deterministic (see Section \ref{ssec:experimental:turn}).

In our case, all comments are synthetic, hence, a comment $c$ in a discussion $d \in D$, at the $i$-th turn is defined recursively as:

\begin{equation}
\label{eq:comment}
\begin{split}
    c(d, i) = & LLM([c(d, j)]^{i-1}_{j=max(1, i-h)};\\
    &\phi(u(d, i)))
\end{split}
\end{equation}

\noindent where $[\cdot]$ is string concatenation and $h$ is the context length of the \ac{LLM} user-agent (how many past comments they can “remember”) and $\phi: U \times M \rightarrow s$ is a function mapping a user $u$ to their instruction prompt $s$.

Our methodology thus assumes that the contents of any synthetic discussion are dependent on the following parameters:
\begin{itemize}[nosep, noitemsep]
    \item The underlying model ($LLM(\cdot)$)
    \item The turn-taking function $u$
    \item The prompting function $\phi$
\end{itemize}


\subsection{Evaluating synthetic discussions}
\label{ssec:methodology:diversity}

As discussed in Section \ref{ssec:related:discussions}, it may not be methodologically sound to attempt approximating realism as the goal of our synthetic discussions. Therefore, we use the “\textit{Diversity}” metric introduced by \citet{ulmer2024} and defined as:

\small
\begin{equation}
\label{eq:variety}
    div(d) = 1- \frac{1}{N(N-1)} \sum_{i=1}^N \sum_{j=1, j \neq i}^N \textit{RLF1}(c(i, d), c(j, d))
\end{equation}
\normalsize

\noindent where \textit{RLF1} is the ROUGE-L F1 score \cite{lin-2004-rouge}. Intuitively, the metric penalizes long, repeated sequences between each pair of comments in a single discussion.  Importantly, this formulation renders the metric invariant to the specific topics discussed, and correlates well with the quality of synthetic data \cite{ulmer2024}.

While maximizing diversity in discussions may seem desirable, it should not be the primary objective, as very high diversity may indicate a lack of meaningful interaction between participants. Instead, we compare the \textit{diversity} distribution of synthetic discussions with that of sampled human discussions. This allows us to estimate the extent to which synthetic discussions approximate real-world ones in terms of content variety and participant interaction. 